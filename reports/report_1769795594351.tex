\documentclass[11pt,letterpaper]{article}

% Essential packages
\usepackage[utf8]{inputenc}
\usepackage[T1]{fontenc}
\usepackage[margin=1in]{geometry}
\usepackage{graphicx}
\usepackage{xcolor}
\usepackage{hyperref}
\usepackage{listings}
\usepackage{float}

% Configure hyperlinks
\hypersetup{
    colorlinks=true,
    linkcolor=blue,
    urlcolor=blue,
    citecolor=blue
}

% Configure code listings
\lstset{
    language=R,
    basicstyle=\ttfamily\small,
    keywordstyle=\color{blue},
    commentstyle=\color{gray}\itshape,
    stringstyle=\color{red},
    numbers=left,
    numberstyle=\tiny\color{gray},
    stepnumber=1,
    numbersep=8pt,
    backgroundcolor=\color{gray!10},
    frame=single,
    breaklines=true,
    breakatwhitespace=false,
    showstringspaces=false,
    xleftmargin=2em,
    framexleftmargin=1.5em
}

\title{Determinants of Adult Obesity Prevalence in the United States}
\date{1/30/2026}
\author{Generated by 3Panel}

\begin{document}
\maketitle

\section{Setup}

\subsection{Required Libraries}

\begin{lstlisting}
library(dplyr)
library(ggplot2)
library(maps)
library(scales)
library(tidyr)
\end{lstlisting}

\subsection{Load Datasets}

\begin{lstlisting}
# Load CSV file (ensure Nutrition_Physical_Activity_and_Obesity.csv is in working directory)
data <- read.csv("Nutrition_Physical_Activity_and_Obesity.csv")

\end{lstlisting}

\section{Educational Attainment and Obesity Prevalence Trends}

Longitudinal analysis demonstrates a persistent inverse correlation between educational attainment and obesity prevalence across the 2011-2023 period. Adults with less than high school education exhibit the highest obesity rates (35-38\%), while college graduates maintain significantly lower rates (25-27\%), representing an approximate 10 percentage point differential that remains stable across all measured years. Notably, all educational strata demonstrate a gradual upward trajectory in obesity prevalence over the observation period, suggesting systemic factors affecting the population regardless of educational status.

\begin{lstlisting}

# Filter for obesity data at national level by education over time
obesity_education_time <- Nutrition_Physical_Activity_and_Obesity %>%
  filter(grepl("obesity", Question, ignore.case = TRUE),
         LocationDesc == "National",
         StratificationCategory1 == "Education",
         YearStart >= 2011,
         YearStart <= 2023,
         !is.na(Data_Value)) %>%
  group_by(YearStart, Stratification1) %>%
  summarise(Avg_Obesity_Rate = mean(Data_Value, na.rm = TRUE), .groups = "drop")

# Reorder education levels logically
education_order <- c("Less than high school", "High school graduate", 
                     "Some college or technical school", "College graduate")

obesity_education_time <- obesity_education_time %>%
  mutate(Stratification1 = factor(Stratification1, levels = education_order))

# Create line plot
ggplot(obesity_education_time, aes(x = YearStart, y = Avg_Obesity_Rate, 
                                    color = Stratification1, group = Stratification1)) +
  geom_line(linewidth = 1.1) +
  geom_point(size = 2) +
  scale_x_continuous(breaks = seq(2011, 2023, by = 2)) +
  scale_color_brewer(palette = "PuOr", name = "Education Level") +
  theme_minimal() +
  theme(legend.position = "right") +
  labs(title = "National Obesity Trends by Education Level (2011-2023)",
       subtitle = "Percent of adults aged 18+ with obesity",
       x = "Year",
       y = "Obesity Rate (%)")
\end{lstlisting}

\begin{figure}[H]
\centering
\includegraphics[width=0.85\textwidth]{plot_1769795594351_0_0.pdf}
\caption{Visualization from Analysis 1}
\end{figure}

\subsection*{Output}

\begin{verbatim}
null device 
          1
\end{verbatim}

\section{Geographic Distribution of State-Level Obesity Rates}

Choropleth visualization of 2023 state-level obesity data reveals pronounced regional heterogeneity in adult obesity prevalence across the continental United States. The Southeastern and Midwestern regions demonstrate elevated obesity rates, with states including West Virginia, Mississippi, and Louisiana exhibiting the highest prevalence. In contrast, Mountain West states (notably Colorado) and Northeastern states (including Massachusetts) report significantly lower obesity rates, indicating geographic clustering of obesity burden that may reflect regional differences in socioeconomic conditions, food environments, or cultural factors.

\begin{lstlisting}

# Filter for obesity data by state for 2023
obesity_state_2023 <- Nutrition_Physical_Activity_and_Obesity %>%
  filter(grepl("obesity", Question, ignore.case = TRUE),
         YearStart == 2023,
         StratificationCategory1 == "Total",
         !LocationAbbr %in% c("US", "GU", "PR", "VI"),  # Exclude territories and national
         !is.na(Data_Value)) %>%
  group_by(LocationDesc) %>%
  summarise(Obesity_Rate = mean(Data_Value, na.rm = TRUE), .groups = "drop") %>%
  mutate(region = tolower(LocationDesc))

# Get US state map data
us_states <- map_data("state")

# Merge obesity data with map data
map_data_merged <- us_states %>%
  left_join(obesity_state_2023, by = "region")

# Create choropleth map
ggplot(map_data_merged, aes(x = long, y = lat, group = group, fill = Obesity_Rate)) +
  geom_polygon(color = "white", linewidth = 0.2) +
  scale_fill_gradientn(colors = c("#2E86AB", "#F6F5AE", "#E94F37"),
                       name = "Obesity Rate (%)",
                       na.value = "grey80") +
  coord_fixed(1.3) +
  theme_void() +
  theme(legend.position = "right",
        plot.title = element_text(hjust = 0.5, face = "bold", size = 14),
        plot.subtitle = element_text(hjust = 0.5, size = 10)) +
  labs(title = "Adult Obesity Rates by State (2023)",
       subtitle = "Percent of adults aged 18+ with obesity")
\end{lstlisting}

\begin{figure}[H]
\centering
\includegraphics[width=0.85\textwidth]{plot_1769795594351_1_0.pdf}
\caption{Visualization from Analysis 2}
\end{figure}

\subsection*{Output}

\begin{verbatim}
null device 
          1
\end{verbatim}

\end{document}
